% AER-Article.tex for AEA last revised 22 June 2011
\documentclass[]{AEA}

% The mathtime package uses a Times font instead of Computer Modern.
% Uncomment the line below if you wish to use the mathtime package:
%\usepackage[cmbold]{mathtime}
% Note that miktex, by default, configures the mathtime package to use commercial fonts
% which you may not have. If you would like to use mathtime but you are seeing error
% messages about missing fonts (mtex.pfb, mtsy.pfb, or rmtmi.pfb) then please see
% the technical support document at http://www.aeaweb.org/templates/technical_support.pdf
% for instructions on fixing this problem.

% Note: you may use either harvard or natbib (but not both) to provide a wider
% variety of citation commands than latex supports natively. See below.

% Uncomment the next line to use the natbib package with bibtex
\usepackage{natbib}

% Uncomment the next line to use the harvard package with bibtex
%\usepackage[abbr]{harvard}

% This command determines the leading (vertical space between lines) in draft mode
% with 1.5 corresponding to "double" spacing.
\draftSpacing{1.5}

\usepackage{hyperref}

\begin{document}

\title{Economic Costs of Terror}
\shortTitle{An Asset Market Approach}
% \author{Author1 and Author2\thanks{Surname1: affiliation1, address1, email1.
% Surname2: affiliation2, address2, email2. Acknowledgements}}


\author{
  Edward Jee\thanks{
  Jee: London School of Economics, \href{mailto:E.M.Jee@lse.ac.uk}{E.M.Jee@lse.ac.uk}.
}
}

\date{\today}
\pubMonth{3}
\pubYear{2018}
\pubVolume{1}
\pubIssue{1}
\JEL{A10, A11}
\Keywords{first keyword, second keyword}

\begin{abstract}
I show that large terror attacks do not have a significant negative
effect on UK asset markets. Furthermore, when I expand my analysis to
include all terror attacks from 1970 to present day I find no evidence
of a negative asset market response. Finally, I find no evidence of
event heterogeneity. That is, target type and location; success of an
attack; number of wounded or killed; weapons used; attack type and media
coverage cannot be used as predictors for asset market responses.
Results are consistent across a range of UK indices and specifications.
\end{abstract}


\maketitle

\pagestyle{plain}

Terror attacks in developed countries have long been studied by
economists as sources of exogenous shocks and events of interest in
their own right. However, comparatively little effort has been expended
exploring responses to the terror distribution outside of extreme tail
events such as 9/11 or London 7/7\footnote{To name a few studies
  concerned with large attacks, see Abadie and Dermisi (2008); Sandler
  and Enders (2008) or Draca, Machin, and Witt (2011). The full terror
  spectrum is studied in Abadie and Gardeazabal (2003); Brodeur
  (Forthcoming) and Enders and Sandler (1991).}. This paper aims to
quantify the individual responses to all attempted terror attacks
occurring in the UK from 1970-2016 and estimate the determinants of
these responses.

The Global Terrorism Database (GTD) (2016) defines terror attacks as the
threatened or actual use of illegal force and violence by a non-state
actor to attain a political, economic, religious, or social goal through
fear, coercion, or intimidation. This definition admits two clear
economic channels through which the consequences of an attack propagate:
the direct effect is comprised of destruction of physical or human
capital, whilst the indirect effect deters investment and consumption
through increased uncertainty or heightened perception of terror risk,
rational or otherwise, as well as second-order effects such as greater
costs incurred due to increased border security or as a catalyst for
overseas intervention to disrupt terrorist organisations.

The UK is a natural candidate for terror analysis as a victim of both
extreme events such as the London 7/7 Bombings as well as a long history
of attacks due to the Troubles. Furthermore, the diverse nature of
terrorism in the UK, with attacks inspired by both Islamic extremism and
secessionist movements, suggests my results are likely to be externally
valid and can be applied across a number of western economies where
social and political discourse concerning terrorism is increasingly
prevalent. Finally, the GTD reports over 3000 terror attacks on UK soil
since 1970 as well as more than 100 attack variables providing ample
data with which to estimate terror responses. The high number of attacks
observed, however, is both a blessing and a curse, particularly when it
comes to estimating event heterogeneity, as many attacks have to be
discarded for fear of overlap and contamination of results. I employ a
range of methods to address this overlap problem however none are
entirely satisfactory.

Throughout the paper as a proxy for terror responses I use daily returns
of equity indices. This approach has a number of advantages over
measuring a more direct economic variable of interest such as GDP or
unemployment. Firstly, major UK indices are characterised by high
liquidity and volumes and hence offer a clear, almost instantaeneous
identification of agent reactions to terror attacks. Secondly, index
prices, and therefore returns, are a function of agents' beliefs about
current and future cash flows and therefore capture any changes in both
direct and indirect effects unlike a stock variable. Finally, index
returns encompass a broad range of firms within an economy and can be
interpreted as a good proxy for the stochastic discount factor used in
many traditional macro models.

I use two methods to estimate index responses to attacks. First, I apply
an event study methodology calculating cumulative abnormal returns
(CARs) under the constant mean return model to quantify the effects of
the five largest terror attacks, measured by a weighted sum of injuries
and fatalities, and find mixed evidence of a significant fall in index
returns in four of the five cases but no significant fall on aggregate.
Then, I adapt Chesney, Reshetar, and Karaman (2011)'s conditional
probability approach within a bayesian setting and fail to conclude that
any of the five attacks cause an extreme or abnormal movement in
returns. Next, I apply the same methodologies to twenty of the UK's
largest attacks, stratified by decade, and again fail to reject the null
of significantly negative responses. Finally, I aggregate the cumulative
abnormal returns to find a terror attack cumulative average abnormal
return (CAAR) which, again, is not significantly different from zero.

Estimating event heterogeneity produces broadly similar results. I
regress event CAR's and event day returns on covariates such as weapons
used; event location; event target; number of killed or wounded; attack
success and a range of other variables using a Laplace prior to perform
Bayesian LASSO in order to overcome concerns of overfitting. Projection
predictive feature selection (Piironen and Vehtari 2017) suggests a
number of variables of interest such as media intensity and number of
injured provide the most predictive power when estimating event returns
but my tests lack the power to conclude that these variables are
different from 0 using 90\% credibility intervals.

Overall, it seems that index returns are resilient to terror attacks and
the effects of terror are negligible, whilst more work is needed to
definitively conclude that there is no event heterogeneity present.

\section{Literature Review}

Sample figure:

\begin{figure}
Figure here.

\caption{Caption for figure below.}
\begin{figurenotes}
Figure notes without optional leadin.
\end{figurenotes}
\begin{figurenotes}[Source]
Figure notes with optional leadin (Source, in this case).
\end{figurenotes}
\end{figure}

Sample table:

\begin{table}
\caption{Caption for table above.}

\begin{tabular}{lll}
& Heading 1 & Heading 2 \\
Row 1 & 1 & 2 \\
Row 2 & 3 & 4%
\end{tabular}
\begin{tablenotes}
Table notes environment without optional leadin.
\end{tablenotes}
\begin{tablenotes}[Source]
Table notes environment with optional leadin (Source, in this case).
\end{tablenotes}
\end{table}

References here (manual or bibTeX). If you are using bibTeX, add your
bib file name in place of BibFile in the bibliography command. \% Remove
or comment out the next two lines if you are not using bibtex.
\bibliographystyle{aea} \bibliography{references}

\% The appendix command is issued once, prior to all appendices, if any.
\appendix

\section{Mathematical Appendix}

\hypertarget{refs}{}
\hypertarget{ref-ABADIE2008451}{}
Abadie, Alberto, and Sofia Dermisi. 2008. ``Is Terrorism Eroding
Agglomeration Economies in Central Business Districts? Lessons from the
Office Real Estate Market in Downtown Chicago.'' \emph{Journal of Urban
Economics} 64 (2): 451--63.
doi:\href{https://doi.org/https://doi.org/10.1016/j.jue.2008.04.002}{https://doi.org/10.1016/j.jue.2008.04.002}.

\hypertarget{ref-10.2307ux2f3132164}{}
Abadie, Alberto, and Javier Gardeazabal. 2003. ``The Economic Costs of
Conflict: A Case Study of the Basque Country.'' \emph{The American
Economic Review} 93 (1). American Economic Association: 113--32.
\url{http://www.jstor.org/stable/3132164}.

\hypertarget{ref-B_forthcoming}{}
Brodeur, Abel. Forthcoming. ``The Effect of Terrorism on Employment and
Consumer Sentiment: Evidence from Successful and Failed Terror
Attacks.'' \emph{American Economic Journal} Forthcoming.
\url{https://www.aeaweb.org/articles?id=10.1257/app.20160556}.

\hypertarget{ref-CHESNEY2011253}{}
Chesney, Marc, Ganna Reshetar, and Mustafa Karaman. 2011. ``The Impact
of Terrorism on Financial Markets: An Empirical Study.'' \emph{Journal
of Banking \& Finance} 35 (2): 253--67.
doi:\href{https://doi.org/https://doi.org/10.1016/j.jbankfin.2010.07.026}{https://doi.org/10.1016/j.jbankfin.2010.07.026}.

\hypertarget{ref-10.1257ux2faer.101.5.2157}{}
Draca, Mirko, Stephen Machin, and Robert Witt. 2011. ``Panic on the
Streets of London: Police, Crime, and the July 2005 Terror Attacks.''
\emph{American Economic Review} 101 (5): 2157--81.
doi:\href{https://doi.org/10.1257/aer.101.5.2157}{10.1257/aer.101.5.2157}.

\hypertarget{ref-doi:10.1080ux2f10576109108435856}{}
Enders, Walter, and Todd Sandler. 1991. ``Causality Between
Transnational Terrorism and Tourism: The Case of Spain.''
\emph{Terrorism} 14 (1). Routledge: 49--58.
doi:\href{https://doi.org/10.1080/10576109108435856}{10.1080/10576109108435856}.

\hypertarget{ref-Piironen2017}{}
Piironen, Juho, and Aki Vehtari. 2017. ``Comparison of Bayesian
Predictive Methods for Model Selection.'' \emph{Statistics and
Computing} 27 (3): 711--35.
doi:\href{https://doi.org/10.1007/s11222-016-9649-y}{10.1007/s11222-016-9649-y}.

\hypertarget{ref-sandler_enders_2008}{}
Sandler, Todd, and Walter Enders. 2008. ``Economic Consequences of
Terrorism in Developed and Developing Countries: An Overview.'' In
\emph{Terrorism, Economic Development, and Political Openness}, edited
by Philip Keefer and NormanEditors Loayza, 17--47. Cambridge University
Press.
doi:\href{https://doi.org/10.1017/CBO9780511754388.002}{10.1017/CBO9780511754388.002}.

\hypertarget{ref-GTD}{}
Study of Terrorism, National Consortium for the, and Responses to
Terrorism (START). 2016. ``Global Terrorism Database.''
\url{https://www.start.umd.edu/gtd}.

\end{document}

