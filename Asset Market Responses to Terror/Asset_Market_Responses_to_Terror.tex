% AER-Article.tex for AEA last revised 22 June 2011
\documentclass[]{AEA}

% The mathtime package uses a Times font instead of Computer Modern.
% Uncomment the line below if you wish to use the mathtime package:
%\usepackage[cmbold]{mathtime}
% Note that miktex, by default, configures the mathtime package to use commercial fonts
% which you may not have. If you would like to use mathtime but you are seeing error
% messages about missing fonts (mtex.pfb, mtsy.pfb, or rmtmi.pfb) then please see
% the technical support document at http://www.aeaweb.org/templates/technical_support.pdf
% for instructions on fixing this problem.

% Note: you may use either harvard or natbib (but not both) to provide a wider
% variety of citation commands than latex supports natively. See below.

% Uncomment the next line to use the natbib package with bibtex
\usepackage{natbib}

% Uncomment the next line to use the harvard package with bibtex
%\usepackage[abbr]{harvard}

% This command determines the leading (vertical space between lines) in draft mode
% with 1.5 corresponding to "double" spacing.
\draftSpacing{1.5}

\usepackage{hyperref}

\begin{document}

\title{Economic Costs of Terror}
\shortTitle{An Asset Market Approach}
% \author{Author1 and Author2\thanks{Surname1: affiliation1, address1, email1.
% Surname2: affiliation2, address2, email2. Acknowledgements}}


\author{
  Edward Jee\thanks{
  Jee: London School of Economics, \href{mailto:E.M.Jee@lse.ac.uk}{E.M.Jee@lse.ac.uk}.
}
}

\date{\today}
\pubMonth{3}
\pubYear{2018}
\pubVolume{1}
\pubIssue{1}
\JEL{A10, A11}
\Keywords{first keyword, second keyword}

\begin{abstract}
I show that large terror attacks do not have a significant negative
effect on UK asset markets. Furthermore, when I expand my analysis to
include all terror attacks from 1970 to present day I find no evidence
of a negative asset market response. Finally, I find no evidence of
event heterogeneity. That is, target type and location; success of an
attack; number of wounded or killed; weapons used; attack type and media
coverage cannot be used as predictors for asset market responses.
Results are consistent across a range of UK indices and specifications.
\end{abstract}


\maketitle

\pagestyle{plain}

Terror attacks in developed countries have long been studied by
economists as sources of exogenous shocks and events of interest in
their own right. However, comparatively little effort has been expended
exploring responses to the terror distribution outside of extreme tail
events such as 9/11 or London 7/7\footnote{To name a few studies
  concerned with large attacks, see Abadie and Dermisi (2008); Sandler
  and Enders (2008) or Draca, Machin, and Witt (2011). The full terror
  spectrum is studied in Abadie and Gardeazabal (2003); Brodeur
  (Forthcoming) and Enders and Sandler (1991).}. This paper aims to
quantify the individual responses to all attempted terror attacks
occurring in the UK from 1970-2016 and estimate the determinants of
these responses.

The Global Terrorism Database (GTD) (2016) defines terror attacks as the
threatened or actual use of illegal force and violence by a non-state
actor to attain a political, economic, religious, or social goal through
fear, coercion, or intimidation. This definition admits two clear
economic channels through which the consequences of an attack propagate:
the direct effect is comprised of destruction of physical or human
capital, whilst the indirect effect deters investment and consumption
through increased uncertainty or heightened perception of terror risk,
rational or otherwise, as well as second-order effects such as greater
costs incurred due to increased border security or as a catalyst for
overseas intervention to disrupt terrorist organisations.

The UK is a natural candidate for terror analysis as a victim of both
extreme events such as the London 7/7 Bombings as well as a long history
of attacks due to the Troubles. Furthermore, the diverse nature of
terrorism in the UK, with attacks inspired by both Islamic extremism and
secessionist movements, suggests my results are likely to be externally
valid and can be applied across a number of western economies where
social and political discourse concerning terrorism is increasingly
prevalent. Finally, the GTD reports over 3000 terror attacks on UK soil
since 1970 as well as more than 100 attack variables providing ample
data with which to estimate terror responses. The high number of attacks
observed, however, is both a blessing and a curse, particularly when
estimating event heterogeneity, as many attacks have to be discarded for
fear of overlap and contamination of results. I employ a range of
methods to address this overlap problem however none are entirely
satisfactory.

Throughout the paper as a proxy for terror responses I use daily equity
index returns. This approach has a number of advantages over measuring a
more direct economic variable of interest such as GDP or unemployment.
Firstly, major UK indices are characterised by high liquidity and
volumes and hence offer a clear, almost instantaeneous identification of
agent reactions to terror attacks. Secondly, index prices, and therefore
returns, are a function of agents' beliefs about current and future cash
flows and therefore capture any changes in both direct and indirect
effects unlike a stock variable. Finally, index returns encompass a
broad range of firms within an economy and can be interpreted as a good
proxy for the stochastic discount factor used in many traditional macro
models.

I use two methods to estimate index responses to attacks. First, I apply
an event study methodology, which is commonly used in the finance
literature\footnote{A 2007 estimate puts the number of published papers
  in finance using an event study methodology at over 565 (Kothari and
  Warner 2007).}, to identify the effect of an event on asset prices or
returns. Event studies are particularly useful in this situation due to
their ability to handle small N, large T situations common to financial
time series and extreme event modelling. I estimate cumulative abnormal
returns (CARs) under the constant mean return model to quantify the
effects of the five largest terror attacks, measured by a weighted sum
of injuries and fatalities, and find mixed evidence of a significant
fall in index returns in four of the five cases but no significant fall
on aggregate. Then, I adapt Chesney, Reshetar, and Karaman (2011)'s
conditional probability approach within a Bayesian setting and fail to
conclude that any of the five attacks cause an extreme or abnormal
movement in returns. Next, I apply the same methodologies to twenty of
the UK's largest attacks, stratified by decade, and again fail to reject
the null of significantly negative responses. Finally, I aggregate the
cumulative abnormal returns to find a terror attack cumulative average
abnormal return (CAAR) which, again, is not significantly different from
zero.

Estimating event heterogeneity produces broadly similar results. I
regress event CARs and event day returns on covariates such as weapons
used; event location; event target; number of killed or wounded; attack
success and a range of other variables using a Laplace prior to perform
Bayesian LASSO in order to overcome concerns of overfitting. Projection
predictive feature selection (Piironen and Vehtari 2017) suggests a
number of variables of interest such as media intensity and number of
injured provide the most predictive power when estimating event returns
but my tests lack the power to conclude that these variables are
different from 0 using 90\% credibility intervals.

Overall, it seems that index returns are resilient to terror attacks and
the effects of terror are negligible, whilst more work is needed to
definitively conclude that there is no event heterogeneity present.

\section{Literature Review}

\section{Methods and Results}\subsection{Event Study}

I use a standard event study approach, outlined by MacKinlay (1997),
where \(\tau\) is defined in relation to event time so that \(\tau = 0\)
indicates the day of the attack. I first calculate abnormal returns at
time \(\tau\) for event \(i\) to give cumulative abnormal returns:

\[ \begin{aligned} AR_{i,\tau} &=R_{i,\tau}-E[R_{i,\tau}\vert\Omega_{i,\tau}] \\
CAR_{i(\tau_{1},\tau_{2})} &=\sum_{t=\tau_{1}}^{\tau_{2}} AR_{i,t} \end{aligned}
\] where \(\tau_1 = 0\) and \(\tau_2 = 9\), indicating an event window
of ten days, and \(E[R_{i,\tau}\vert\Omega_{i,\tau}]\) is the expected
index return derived from the constant mean return model conditioning on
information, \(\Omega_{i,\tau}\), common to all agents. The constant
mean return model can be written as:
\[ \begin{aligned} R_{it} &= \mu_i + \psi_{it} \\ E(\psi_{it}) &= 0, \ \text{var}(\psi_{it}) = \sigma_{\psi_{it}}^2 \end{aligned}\]

where \(\mu_i\) is some constant and \(\psi_{it}\) white noise
i.e.~index prices follow a random walk with drift\footnote{From log
  differencing prices \(P_{it}\) to get returns \(R_{it}\).}. Whilst the
constant mean return model is the most simple,\footnote{Compared to more
  advanced methods such as the market model which incorporates CAPM or
  multi-factor models building on Fama and French (1993)'s three factor
  model.} it has been shown by Brown and Warner (1980) to perform
similar to more advanced methods and has the advantage of being well
defined for indices in addition to individual securities which would be
problematic under a CAPM approach. To estimate constant mean returns I
use a twenty day estimation window that ends ten days before the terror
attack. Unfortunately, there is a trade-off unique to event studies
between T and N; a longer estimation window, i.e.~greater T, results in
less events being included in the experiment as there is greater risk of
event overlap i.e.~another event ocurring during the estimation (or
event) window. This problem becomes particularly acute when studying
many thousands of small terror attacks rather than one catastrophic
event.\footnote{Particularly if attacks are planned as part of a terror
  campaign over the course of multiple days.} To overcome this problem I
report both screened and overlapping results where appropriate and
perform robustness checks using a constant median return model which
should be less sensitive to potential overlap issues.

After calculating CARs for each event I aggregate estimates into
cumulative average abnormal returns (CAARs):

\[CAAR_{(\tau_{1},\tau_{2})}=\frac{1}N \sum_{i=1}^{N} CAR_{i(\tau_{1},\tau_{2})}\]
it is these CAARs that form our estimators of interest since they rely
on fewer identifying assumptions than the CARs they are are derived
from.

For CAARs and CARs to admit a causal interpretation we need to rely on
the following assumption: \(E(\textit{attack}_i | \Omega_{i\tau}) = 0\).
That is, terror attacks are realised orthogonally to any current and
past information that influences equity indices. It's worth noting that
this structure allows for terror attacks that aren't independent and
identically distributed provided agents can surmise this from the
(first) attack, since the estimation window is re-estimated using a new
information set, \(\Omega_{i\tau}\), for each event. However, in
practice we need i.i.d. events\footnote{I attempt to relax this
  assumption in the conditional probability approach to little success.}
because I screen for overlapping attacks. For example, any events that
are planned to occur in close proximity as part of a terror campaign
will therefore be systematically removed from the dataset - events are
not dropped at random.

Finally, for CARs to be interpreted causally we also need to make use of
a `tweaked' parallel trends assumption where the estimated constant mean
return \(E[R_{i,\tau}\vert\Omega_{i,\tau}] = \bar{\mu}\) is identical
for event \(i\)'s estimation \textit{and} event window. Essentially
\(E[R_{i,\tau}\vert\Omega_{i,\tau}]\) needs to be a good counterfactual
for stock market returns in the absence of a terror attack. However, our
estimator of interest, CAARs, can relax this assumption by exploiting
the law of large numbers\footnote{Whether five or twenty events are
  enough for this is debatable.} and rely instead on any deviations from
the parallel trends assumption to be random and hence wash out during
aggregation.

\section{References}\bibliographystyle{aea}\bibliography{BibFile.bib}

\appendix

\hypertarget{refs}{}
\hypertarget{ref-ABADIE2008451}{}
Abadie, Alberto, and Sofia Dermisi. 2008. ``Is Terrorism Eroding
Agglomeration Economies in Central Business Districts? Lessons from the
Office Real Estate Market in Downtown Chicago.'' \emph{Journal of Urban
Economics} 64 (2): 451--63.
doi:\href{https://doi.org/https://doi.org/10.1016/j.jue.2008.04.002}{https://doi.org/10.1016/j.jue.2008.04.002}.

\hypertarget{ref-10.2307ux2f3132164}{}
Abadie, Alberto, and Javier Gardeazabal. 2003. ``The Economic Costs of
Conflict: A Case Study of the Basque Country.'' \emph{The American
Economic Review} 93 (1). American Economic Association: 113--32.
\url{http://www.jstor.org/stable/3132164}.

\hypertarget{ref-B_forthcoming}{}
Brodeur, Abel. Forthcoming. ``The Effect of Terrorism on Employment and
Consumer Sentiment: Evidence from Successful and Failed Terror
Attacks.'' \emph{American Economic Journal} Forthcoming.
\url{https://www.aeaweb.org/articles?id=10.1257/app.20160556}.

\hypertarget{ref-BROWN1980205}{}
Brown, Stephen J., and Jerold B. Warner. 1980. ``Measuring Security
Price Performance.'' \emph{Journal of Financial Economics} 8 (3):
205--58.
doi:\href{https://doi.org/https://doi.org/10.1016/0304-405X(80)90002-1}{https://doi.org/10.1016/0304-405X(80)90002-1}.

\hypertarget{ref-CHESNEY2011253}{}
Chesney, Marc, Ganna Reshetar, and Mustafa Karaman. 2011. ``The Impact
of Terrorism on Financial Markets: An Empirical Study.'' \emph{Journal
of Banking \& Finance} 35 (2): 253--67.
doi:\href{https://doi.org/https://doi.org/10.1016/j.jbankfin.2010.07.026}{https://doi.org/10.1016/j.jbankfin.2010.07.026}.

\hypertarget{ref-10.1257ux2faer.101.5.2157}{}
Draca, Mirko, Stephen Machin, and Robert Witt. 2011. ``Panic on the
Streets of London: Police, Crime, and the July 2005 Terror Attacks.''
\emph{American Economic Review} 101 (5): 2157--81.
doi:\href{https://doi.org/10.1257/aer.101.5.2157}{10.1257/aer.101.5.2157}.

\hypertarget{ref-doi:10.1080ux2f10576109108435856}{}
Enders, Walter, and Todd Sandler. 1991. ``Causality Between
Transnational Terrorism and Tourism: The Case of Spain.''
\emph{Terrorism} 14 (1). Routledge: 49--58.
doi:\href{https://doi.org/10.1080/10576109108435856}{10.1080/10576109108435856}.

\hypertarget{ref-FAMA19933}{}
Fama, Eugene F., and Kenneth R. French. 1993. ``Common Risk Factors in
the Returns on Stocks and Bonds.'' \emph{Journal of Financial Economics}
33 (1): 3--56.
doi:\href{https://doi.org/https://doi.org/10.1016/0304-405X(93)90023-5}{https://doi.org/10.1016/0304-405X(93)90023-5}.

\hypertarget{ref-Kothari20073}{}
Kothari, S.P., and Jerold B. Warner. 2007. ``Chapter 1 - Econometrics of
Event Studies*.'' In \emph{Handbook of Empirical Corporate Finance},
edited by B. Espen Eckbo, 3--36. Handbooks in Finance. San Diego:
Elsevier.
doi:\href{https://doi.org/https://doi.org/10.1016/B978-0-444-53265-7.50015-9}{https://doi.org/10.1016/B978-0-444-53265-7.50015-9}.

\hypertarget{ref-10.2307ux2f2729691}{}
MacKinlay, A. Craig. 1997. ``Event Studies in Economics and Finance.''
\emph{Journal of Economic Literature} 35 (1). American Economic
Association: 13--39. \url{http://www.jstor.org/stable/2729691}.

\hypertarget{ref-Piironen2017}{}
Piironen, Juho, and Aki Vehtari. 2017. ``Comparison of Bayesian
Predictive Methods for Model Selection.'' \emph{Statistics and
Computing} 27 (3): 711--35.
doi:\href{https://doi.org/10.1007/s11222-016-9649-y}{10.1007/s11222-016-9649-y}.

\hypertarget{ref-sandler_enders_2008}{}
Sandler, Todd, and Walter Enders. 2008. ``Economic Consequences of
Terrorism in Developed and Developing Countries: An Overview.'' In
\emph{Terrorism, Economic Development, and Political Openness}, edited
by Philip Keefer and NormanEditors Loayza, 17--47. Cambridge University
Press.
doi:\href{https://doi.org/10.1017/CBO9780511754388.002}{10.1017/CBO9780511754388.002}.

\hypertarget{ref-GTD}{}
Study of Terrorism, National Consortium for the, and Responses to
Terrorism (START). 2016. ``Global Terrorism Database.''
\url{https://www.start.umd.edu/gtd}.

\end{document}

